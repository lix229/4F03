% Created 2020-01-07 Tue 14:11
% Intended LaTeX compiler: pdflatex
\documentclass[11pt]{article}
\usepackage[utf8]{inputenc}
\usepackage[T1]{fontenc}
\usepackage{graphicx}
\usepackage{grffile}
\usepackage{longtable}
\usepackage{wrapfig}
\usepackage{rotating}
\usepackage[normalem]{ulem}
\usepackage{amsmath}
\usepackage{textcomp}
\usepackage{amssymb}
\usepackage{capt-of}
\usepackage{hyperref}
\author{Shawn Li}
\date{\today}
\title{COMPSCI 4F03 Class Notes}
\hypersetup{
 pdfauthor={Shawn Li},
 pdftitle={COMPSCI 4F03 Class Notes},
 pdfkeywords={},
 pdfsubject={},
 pdfcreator={Emacs 26.3 (Org mode 9.2.6)}, 
 pdflang={English}}
\begin{document}

\maketitle
\tableofcontents


\section{Introduction}
\label{sec:orgd2d918e}
\subsection{How to handle a large complex task?}
\label{sec:orgbe5ac55}
\subsubsection{Sequential}
\label{sec:org0bee5ae}
One person works on different parts of the task.
\subsubsection{Parallel}
\label{sec:org5f3d317}
A number of person work on different parts, that are combined to complete the task.
\begin{enumerate}
\item Advantages of second approach
\label{sec:orgb7f0917}
\begin{enumerate}
\item Same job in less time.
\item Much bigger jobs can be handled in reasonable time.
\end{enumerate}
\item Presumptions of second approach
\label{sec:org2b48037}
\begin{enumerate}
\item There will be contension.
\item Because of the contension, there have to be communication.
\item Consequently, in real world, the ideal result would be not be possible, because communication
and resolving contension will need time, however, the improvements are significent.
\item Diminishingr returns: as more workers/threads are added, the benifit of each worker/thread added
will be reducing, to a certain point the benifits are almost zero.
\end{enumerate}
\end{enumerate}
\subsubsection{Distributed parallelism}
\label{sec:org92a86b2}
Separate memories to reduce contension, but communication will be increasing.
\begin{enumerate}
\item Load balancing
\label{sec:orge86e239}
Work load should be roughly the same for each worker/processors, but communication is still necessary.
\end{enumerate}
\subsection{Parallel Programming}
\label{sec:org835798f}
Bigger problems can be solved if more than one computers are used in parallel.
\begin{enumerate}
\item Sequential programming will not work. Mature tools and compilers are not available.
\item Locations of processors can be different.
\begin{enumerate}
\item Inside a single enclosure.
\item A number of computers connected together.
\end{enumerate}
\item Different interconnection methods can be used.
\begin{enumerate}
\item Direct connections.
\item Ethernet.
\end{enumerate}
\end{enumerate}
\subsubsection{Alternatives}
\label{sec:orgfbe957b}
Very fast single processor computers with large memory bandwith. It is more familiar with existing technologies.
But it is very expensive, gives out a lot of
 heat. Also, the performance of a single CPU eventually will reach its limit.
\subsubsection{Software considerations}
\label{sec:org9e8ef46}
\begin{enumerate}
\item How to decompose a program into seperate independent parts.
\item Parrellel algorithms.
\item Operating system support.
\end{enumerate}
\subsubsection{Different parallel programming models.}
\label{sec:orgb3e65f3}
\begin{enumerate}
\item Shared memory
\item Message passing
\item Instruction level parallelism
\item Multithreading
\item Data parallelism
\item Hybrid systems
\end{enumerate}
\end{document}
