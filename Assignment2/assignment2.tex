% Created 2020-02-09 Sun 22:27
% Intended LaTeX compiler: pdflatex
\documentclass[11pt]{article}
\usepackage[utf8]{inputenc}
\usepackage[T1]{fontenc}
\usepackage{graphicx}
\usepackage{grffile}
\usepackage{longtable}
\usepackage{wrapfig}
\usepackage{rotating}
\usepackage[normalem]{ulem}
\usepackage{amsmath}
\usepackage{textcomp}
\usepackage{amssymb}
\usepackage{capt-of}
\usepackage{hyperref}
\author{Shawn Li, lix229}
\date{\today}
\title{CS4F03 ASSIGNMENT 2}
\hypersetup{
 pdfauthor={Shawn Li, lix229},
 pdftitle={CS4F03 ASSIGNMENT 2},
 pdfkeywords={},
 pdfsubject={},
 pdfcreator={Emacs 26.3 (Org mode 9.3.2)}, 
 pdflang={English}}
\begin{document}

\maketitle
\tableofcontents

\newpage

\section{Question 1}
\label{sec:org75975a8}

\subsection{Part A}
\label{sec:org3b25f8b}
\emph{2-D mesh}:\\
 	diameter: 2 \texttimes{} (\sqrt{36}-1) = 10 \\
 	width: \sqrt{36} = 6 \\
\emph{2-D torus}: \\
	diameter: 2 \texttimes{} (\sqrt{36}/2) = 6 \\
	width: 2 \texttimes{} \sqrt{36} = 12\\
\emph{6-D cube} : \\
	diameter: 6\\
	width: 18\\

The 6-D hypercube is the best alternative, since when we design a high performance multicomputer platform, we expect lower diameter and higher width so that the parallel computer can require communication  between pairs of nodes with costing less time and require a large amount of data at one time.

\subsection{Part B}
\label{sec:org906425f}
The current address X is 101010 for nodes 42 and the destination address Y is 001101 for node 13.
\begin{enumerate}
\item Exclusive OR: 100111
\item The most significent 1-bit: 6th bit.
\item Negate the 6th bit address: 001010.
\item Exclusive OR: 000111
\item The most significent 1-bit: 3rd bit
\item Negate the 3rd bit: 001110
\item Exclusive OR: 000011
\item The most significent 1-bit: 2nd bit
\item Negate the 2nd bit: 001100
\item Exclusive OR: 000001
\item The most siginicent 1-bit: 1st bit
\item Exclusive OR: 001101
\end{enumerate}

\section{Question 2}
\label{sec:orgf755713}
\subsection{Data size varies}
\label{sec:org5229c38}
s + np = 0.12 + 0.88 \texttimes{} 56 = 49.4 \\
49.4 \texttimes{} 2.5 = 123.5Gflops
\subsection{Data size fixed}
\label{sec:org7ca0f5c}
1/(0.12 + 0.888/n) = n/(0.12n + 0.88) =  7.36
7.36 \texttimes{} 2.5 = 18.42Gflops

\section{Question 3}
\label{sec:orgc36ad8c}
\begin{enumerate}
\item 00 output port0 of  A1
\item 10 output  port2 of B1
\item 11 output port port3 of C2
\item Arrive at Node 9
\end{enumerate}
\end{document}
